%\chapter[Naslov poglavlja u sadržaju][Kratki naslov poglavlja]{Naslov poglavlja}	
% ukoliko naslov nije jako dugacak dovoljno je samo \chapter{Naslov poglavlja} 

%\section[Naslov sekcije u sadržaju][Kratki naslov sekcije]{Naslov sekcije}
%\subsection{Naslov podsekcije}
%\begin{thm}
%Iskaz teorema u kojem se javljaju skupovi  $\N$, $\Z$, $\Q$, $\R$ i $\C$.
%\end{thm}
%\begin{conj}
%Iskaz slutnje u kojoj se javljaju funkcije $\tg$, $\tgh$ i $\sh$.
%\end{conj}
%\begin{cor}
%Iskaz posljedice u kojoj se javljaju skupovi $\Ker T$ i $\slika T$..
%\end{cor}
%\begin{proof}
%Dokaz posljedice se nalazi u \cite{kljuc}. Pogledajte i \cite{kurepa1956convex}, \cite{kurepa1981funkcionalna} te \cite{Dutkay:2009}.
%\end{proof}SDGFSFDG
% Dodatno definirane matematicke okoline:
% teorem (okolina: thm), lema (okolina: lem), korolar (okolina: cor),
% propozicija (okolina: prop), definicija (okolina: defn), napomena (okolina: rem),
% slutnja (okolina: conj), primjer (okolina: exa), dokaz (okolina: proof)
% Definirane su naredbe za ispisivanje skupova N, Z, Q, R i C
% Definirane su naredbe za funkcije koje se u hrvatskoj notaciji oznacavaju drukcije 
% nego u americkoj: tg, ctg, ... (\tgh za tangens hiperbolni)
% Takodjer su definirane naredbe za Ker i Im (da bi se razlikovala od naredbe za imaginarni dio kompleksnog
% broja, naredba se zove \slika).

\begin{thm}
Iskaz teorema u kojem se javljaju skupovi  $\N$, $\Z$, $\Q$, $\R$ i $\C$.
\end{thm}
Usmjeren težinski graf je dan kao trojka $\mathcal{G} = ( \mathcal{V}, \mathcal{E}, \{ \mathcal{w}_{j,k} \}_{j,k=1}^{n} )$, gdje je $\mathcal{V} = \{v_1,...,v_n\}$ neprazan skup vrhova, $\mathcal{E} \subset \mathcal{V} \times \mathcal{V} $ skup bridova i $\mathcal{w}_{j,k} \geq 0 $ su težine bridova. $\mathcal{w}_{j,k} > 0$  akko $(v_j,v_k) \in \mathcal{E}$. Za usmjeren graf je općenito: $\mathcal{w}_{j,k} \neq \mathcal{w}_{k,j}$ 

Napomena
Nije nužno $\mathcal{w}_{j,k} = \mathcal{w}_{k,j}$.
wjk = wkj for all j, k and wjk > 0
iff (vj, vk) 